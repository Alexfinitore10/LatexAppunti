\documentclass[11pt]{article}
\newtheorem{theorem}{Theorem}


\title{La Fluido Dinamica}

\date{\today}


\begin{document}

\maketitle

Su un fluido in \textbf{quiete} agiscono 2 forze:

\begin{itemize}
\item Forze di Volume
\item Forze di Superficie
\end{itemize}

In stato di quiete tutte le forze sono equivalenti.
$$\sum_{i} F_n = 0$$
Le forze di superficie sono uguali e opposte e quindi la loro risultante e` 0.

% grafico qui

$$\sum F_y = 0$$

$$\sum F_2 - F_1 - F_p = 0$$

$$P_2 \cdot A - P_1 \cdot A - \rho Vg = 0$$

In questo caso possiamo considerare $V = Ah$

$$P_2 - P_1 - \rho hg$$

Da qui possiamo ricavare la formula e conseguente formalizzata come \textbf{Legge di Stevino}

\begin{theorem}[Legge Di Stevino]
Questa legge stabilisce che la pressione totale impressa dal fluido, dipende solo dall'altezza (cioe la profondita).
\[P_2 = P_1 + \rho hg\]
\end{theorem}

\newpage

\begin{Large}
Il Principio di Pascal

\end{Large}

Ogni variazione di pressione applicata a un fluido chiuso viene trasmessa ad ogni punto del fluido e a ogni parete.

Un esempio di applicazione di questo principio e la \textbf{la pressa idraulica}.

%foto grafico pressa idraulica

\end{document}